\documentclass[12pt]{article}
\usepackage{amsmath}
\usepackage{amsfonts}
\usepackage{braket}
% \usepackage[utf8]{inputenc}
\newcommand{\angstrom}{\mbox{\normalfont\AA}}
\begin{document}

\title{Discrete Variable Representation Introduction and Derivation}
\author{Luke Hatcher}
\date{University of Washington \\ McCoy Group, Winter 2020}
\maketitle


\section{Discrete Variable Representation}
\subsection{Introduction}
Discrete Variable Representation (DVR) is an algorithmic approach to solving the time independent Schrodinger equation. It is a commonly used tool in the field of computational chemistry. 
\subsection{Basis functions}
For DVR, a finite set of basis functions must be chosen. 
For this case our set $\{\phi_i\}_N$ belongs to a space such that they satisfy
\begin{equation}
\braket{\phi_i|\phi_j} = \delta_{ij}
\end{equation}
and thus are orthonormal.  
It follows that we can represent our wavefunction as
\begin{equation}
\psi=\sum_i{c_i}\phi_i
\end{equation}

\begin{equation}
c_i = \braket{\phi_i|\psi}
\end{equation}

\subsection{Matrix Representations}
Starting from the time independent Schrodinger equation
\begin{equation}
\hat{H}\psi = E\psi
\end{equation}
recall that
\begin{equation}
\hat{H}=\hat{T}+\hat{V}
\end{equation}
We want to represent our Hamiltonian as the sum of two matrices.
Thanks to Colbert and Miller, the matrix elements of the kinetic energy matrix are analytically known.$^{[1]}$
\begin{equation}
\hat{\textbf{T}}_{ij} = \frac{\hbar^2(-1)^{i-j}}{2m\Delta{x^2}}
\begin{cases}
\frac{\pi^2}{3}, &i=j\\
\frac{2}{(i-j)^2}, &i\neq{j}
\end{cases}
\end{equation}
The potential at each point $V(x_i)$ must be evaluated using a potential energy surface for the system being studied. 
t follows that the potential energy matrix is diagonal.
\begin{equation}
\hat{\textbf{V}}_{ij}=\delta_{ij}V(x_i)
\end{equation}
Here the grid points $x_i$ are evenly spaced by $\Delta{x}$ and the size of the grid is appropriately large so that the wavefunction tends to zero at both ends.
Doing so will represent an interval of $(-\infty, \infty)$.
Now, we can add our kinetic and potential matrix representations together to get our Hamiltonian matrix.
\begin{equation}
\hat{\textbf{H}}=\hat{\textbf{T}}+\hat{\textbf{V}}
\end{equation}

\subsection{Solving for $\psi$}
Now all that remains is an eigenvalue problem.
\begin{equation}
\hat{\textbf{H}}\psi=\epsilon_n\psi
\end{equation}

\begin{equation}
(\hat{\textbf{H}}-\epsilon_\lambda\textbf{I})\psi=0
\end{equation}
Nontrivial solutions exist when 
\begin{equation}
det(\hat{\textbf{H}}-\epsilon_\lambda\textbf{I})=0
\end{equation}
which after solving provides us with our desired energy eigenvalues $\epsilon_n$ and thus their respected eigenvectors $\psi$.

\section{References}
[1] D. T. Colbert and W. H. Miller, J. Chem. Phys. 96, 1982 (1992).




\end{document}



